% !TEX TS-program = xelatex
% !TEX encoding = UTF-8 Unicode
% !Mode:: "TeX:UTF-8"

\documentclass{resume}
\usepackage{zh_CN-Adobefonts_external} % Simplified Chinese Support using external fonts (./fonts/zh_CN-Adobe/)
%\usepackage{zh_CN-Adobefonts_internal} % Simplified Chinese Support using system fonts
\usepackage{linespacing_fix} % disable extra space before next section
\usepackage{cite}

\begin{document}
\pagenumbering{gobble} % suppress displaying page number

\name{徐业成}

\basicInfo{
  \email{doublefisher06@gmail.com} \textperiodcentered\ 
  \phone{(+86) 15270081868} \textperiodcentered\ 
  \linkedin[GitHub:https://github.com/hitxyc]{https://github.com/hitxyc}}
 
\section{\faGraduationCap\  教育背景}
\datedsubsection{\textbf{哈尔滨工业大学(威海)}, 山东, 威海}{2023.09 -- 2027.06}
\textit{在读生}\ 计算机科学与技术

\section{\faUsers\ 实习/项目经历}
\datedsubsection{\textbf{学生管理系统} }{2025年1月}
\role{Golang}{个人项目}
项目描述:使用Go语言实现一个学生信息和成绩管理系统,主要功能包括学生信息的增删改查,使用内存存储数据,不涉及数据库的持久化操作。\\
技术栈:Gin、并发控制、反射
\begin{itemize}
  \item 数据结构:map, slice
  \item 实现了增删改查, 可进行精准查询和分页查询
  \item 支持从CSV文件中批量导入学生信息,并实现并发处理和错误处理机制。
  \item 通过Gin框架搭建HTTP Web服务。
\end{itemize}

\datedsubsection{\textbf{EchoDB(回响录)}}{2025年2月}
\role{Golang}{个人项目}
\begin{onehalfspacing}
项目简介:使用 Go 语言实现一个分布式内存数据库,数据库采用 NoSQL 类型,支持相同实例下的多实例业务。\\
技术栈:Gin、GORM、Redis、MySQL、Raft
\begin{itemize}
 \item 使用 Go的goroutine 和 select语句,实现了异步任务处理和错误回调机制。
  \item 分布式一致性保障:使用 Raft 协议在多个节点间实现数据一致性,保障系统的高可用性和容错性。
  \item 缓存机制:结合 Redis 和懒加载策略对热点数据进行缓存预热,避免了缓存穿透和缓存击穿问题。
\end{itemize}
\end{onehalfspacing}


% Reference Test
%\datedsubsection{\textbf{Paper Title\cite{zaharia2012resilient}}}{May. 2015}
%An xxx optimized for xxx\cite{verma2015large}
%\begin{itemize}
%  \item main contribution
%\end{itemize}

\section{\faCogs\ IT 技能}
% increase linespacing [parsep=0.5ex]
\begin{itemize}[parsep=0.5ex]
  \item Golang基础:熟悉Go语言基础,并发模型,错误处理,网络编程、反射
  \item 数据库:掌握 MySQL 和 GORM,能够根据项目要求设计并实现数据存储方案,了解Redis
  \item API设计与网络编程:擅长使用 Gin 框架进行 RESTful API 的设计和开发
  \item 分布式系统与一致性算法:了解Raft协议
\end{itemize}

\section{\faInfo\ 其他}
% increase linespacing [parsep=0.5ex]
\begin{itemize}[parsep=0.5ex]
  \item GitHub: https://github.com/hitxyc
  \item 语言: 英语 - 熟练(六级 556)
\end{itemize}

%% Reference
%\newpage
%\bibliographystyle{IEEETran}
%\bibliography{mycite}
\end{document}
